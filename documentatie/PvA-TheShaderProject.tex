\documentclass[]{report}
\usepackage[parfill]{parskip}

% Title Page
\title{Plan van Aanpak - TheShaderProject}
\author{Mike Wierenga / Timo Strating}
\date{15 mei 2017, Groningen}

\renewcommand*\contentsname{Table of Content}
\pagestyle{headings}

\begin{document}
\maketitle

\tableofcontents
\newpage






\chapter{Achtergrond}

Het Alfa-College heeft aan een deel van zijn leerlingen voorgelegd om een verdiepingsonderzoek te doen in een ICT onderwerp. Hieruit is het onderwerp shaders naar voren gekomen. Vanuit dit onderwerp is uiteindelijk een Project groep gevormd. 

Na een lange overweging voor een goede repository naam is uiteindelijk de code naam voor dit project bedacht. Het Project zal luisteren naar TheShaderProject.

De repository zal publiekelijk beschikbaar zijn en is te vinden via deze URL:
\begin{center}
	https://github.com/timostrating/TheShaderProject
\end{center}



\chapter{Project opdracht}

Verdieping / onderzoek van de materie omtrent shaders. De kennis die vergaat zal worden in dit onderzoek zal daarna in de form van een samengevatte presentatie worden overgedragen aan mede studenten. 

De andere studenten zullen dit ook andersom gaan doen. Het luisteren en actief meedenken bij andere projecten is daarom dus ook een onderdeel van dit project.




\chapter{Project activiteiten}

\begin{itemize}
	\item Face 1: Ontwerp
	\begin{enumerate}
		\item Het maken van een PvA.
		\item Basis kennis opdoen over de materie.
			\newline
	\end{enumerate} 
	
	\item Face 2: Realisatie
	\begin{enumerate}
		\item Het onderzoek doen naar Shaders.
		\item Het implementeren van de kennis door een van de effecten uit een bekend spelletje na te maken.
			\newline
	\end{enumerate} 
	
	\item Face 3: Inplementatie
	\begin{enumerate}
		\item Het geven van een presentatie.
		\newline
	\end{enumerate}

	\item Face 4: Onderhoud
	\begin{enumerate}
		\item Het bij houden van de repository op Github.
		\newline
	\end{enumerate}
\end{itemize} 





\chapter{Project grenzen}

\section{Lengte van het project}
Er zijn 7 weken de tijd om dit project af te ronden. Binnen deze tijd zal de presentatie ook plaats gaan vinden. Er Zal daarom rekening worden gehouden met 5 weken aan onderzoek.

\section{Breedte van het project}
Er zijn door het Alfa-College enkele uren beschikbaar gesteld om aan dit project te werken. Daarnaast zal er in de avond uren ook aan dit project worden gewerkt. De tijden buiten de beschikbaar gemaakte tijden zullen ongeveer 1 a 2 avonden per week bedragen. 

\section{Deadline}
de deadline zal het moment zijn dat de presentatie gegeven zal worden. Deze datum zal in de aankomende weken worden vastgesteld door het Alfa-College. Er wordt rekening gehouden met een deadline van minimaal 5 weken en maximaal 7 weken.





\chapter{Producten}

\begin{itemize}
	\item Plan van Aanpak (PvA)
	\item Presentaties
	\item Demonstratie van enkele shaders
	\item Git repository gevuld met gemaakte shaders
		\newline
\end{itemize} 





\chapter{Kwaliteit}

\section{Tussenproducten}
Er zal geen tot weinig controle zijn op alles wat als tassentijdsproduct wordt gezien. Dit wordt gedaan om zo snel mogelijk te itereren als project groep.

\section{Eindproduct}	
De presentatie en demonstratie die als eindproduct opgeleverd zal gaan worden zullen in verhouding staan tot alle andere presentatie van de andere project groepen. 

\section{Controle}	
De team leden zullen naar elkaar een controlerende rol krijgen. Daarnaast zal het Alfa-College het project gaan begeleiden. 

\section{Code}		
De Code richtlijnen van Nvidia en Unity zullen gecombineerd worden tot de kwaliteitsrichtlijnen van dit project.




\chapter{Projectorganisatie}

\section{ontwikkel team}
Mike Wierenga \\ \\
\begin{tabular}{ l c }
	mail & m.wierenga@student.alfa-college.nl  \\
\end{tabular}

\vspace*{25px}

Timo Strating \\ \\
\begin{tabular}{ l c }
	mail & t.strating@student.alfa-college.nl  \\
\end{tabular}

\vspace*{25px}


\section{Informatie / communicatie}

Communicatie binnen het team zal voornamelijk via Whatsapp plaats gaan vinden. Github zal gebruikt worden om alle bestanden en gegevens te documenteren. De rest van de communicatie zal plaats vinden op de werkvloer als het ware.



\chapter{Planning}

\begin{itemize}
	\item Face 1: Ontwerp - (2 weken)	
	\item Face 2: Realisatie - (4 weken)	
	\item Face 3: Implementatie  - (1 week)	
	\item Face 4: Onderhoud 
		\newline
\end{itemize} 

Zie Project Activiteiten voor meer informatie




\chapter{Kosten en baten}

Er zijn geen kosten en baten verbonden aan dit project.



\chapter{Risico’s}

Er zijn geen Risico's verbonden aan dit project. Het project is namelijk opgezet als leeromgeving.



\end{document}      
